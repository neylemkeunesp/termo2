% $Header: /cvsroot/latex-beamer/latex-beamer/examples/beamerexample5.tex,v 1.22 2004/10/08 14:02:33 tantau Exp $

\documentclass[11pt]{beamer}

\usetheme{Darmstadt}

\usepackage{times}
\usefonttheme{structurebold}

%\usepackage[english]{babel}
\usepackage[brazilian]{babel}
\usepackage{pgf,pgfarrows,pgfnodes,pgfautomata,pgfheaps}
\usepackage{amsmath,amssymb}
\usepackage[utf8]{inputenc}
%\usepackage[latin1]{inputenc}
\usepackage{graphicx}
\usepackage{tikz}
\usepackage{cancel}
\usetikzlibrary{arrows.meta, positioning, calc}
\setbeamercovered{dynamic}

\newcommand{\Lang}[1]{\operatorname{\text{\textsc{#1}}}}

\newcommand{\Class}[1]{\operatorname{\mathchoice
  {\text{\sf \small #1}}
  {\text{\sf \small #1}}
  {\text{\sf #1}}
  {\text{\sf #1}}}}

\newcommand{\NumSAT}      {\text{\small\#SAT}}
\newcommand{\NumA}        {\#_{\!A}}

\newcommand{\barA}        {\,\bar{\!A}}

\newcommand{\Nat}{\mathbb{N}}
\newcommand{\Set}[1]{\{#1\}}

\pgfdeclaremask{tu}{beamer-tu-logo-mask}
\pgfdeclaremask{computer}{beamer-computer-mask}
\pgfdeclareimage[interpolate=true,mask=computer,height=2cm]{computerimage}{beamer-computer}
\pgfdeclareimage[interpolate=true,mask=computer,height=2cm]{computerworkingimage}{beamer-computerred}
\pgfdeclareimage[mask=tu,height=.5cm]{logo}{logounesp}

\logo{\pgfuseimage{logo}}

\colorlet{redshaded}{red!25!bg}
\colorlet{shaded}{black!25!bg}
\colorlet{shadedshaded}{black!10!bg}
\colorlet{blackshaded}{black!40!bg}

\colorlet{darkred}{red!80!black}
\colorlet{darkblue}{blue!80!black}
\colorlet{darkgreen}{green!80!black}
\definecolor{lightblue}{RGB}{120, 200, 230}
\definecolor{darkblue}{RGB}{30, 100, 130}
\definecolor{darkgray}{RGB}{70, 70, 70}
\definecolor{lightgray}{RGB}{150, 150, 150}

\def\radius{0.96cm}
\def\innerradius{0.85cm}

\def\softness{0.4}
\definecolor{softred}{rgb}{1,\softness,\softness}
\definecolor{softgreen}{rgb}{\softness,1,\softness}
\definecolor{softblue}{rgb}{\softness,\softness,1}

\definecolor{softrg}{rgb}{1,1,\softness}
\definecolor{softrb}{rgb}{1,\softness,1}
\definecolor{softgb}{rgb}{\softness,1,1}

\title{Técnica dos Multiplicadores de Lagrange}
\author{Ney Lemke}
\institute[IBB-UNESP]{%
    Departamento de Biofísica e Farmacologia}
\date{\today}

%\AtBeginSection[]{\frame{\frametitle{Outline}\tableofcontents[current]}}

\begin{document}

\frame{\titlepage}

\part{Multiplicadores de Lagrange}
\frame{\frametitle{Outline}\tableofcontents[part=1]}

\section{Introdução}

\begin{frame}
\frametitle{Introdução}
\begin{itemize}
    \item Os multiplicadores de Lagrange são uma técnica poderosa para otimização com restrições
    \item Desenvolvida por Joseph-Louis Lagrange no século XVIII
    \item Amplamente utilizada em física, engenharia e economia
\end{itemize}
\end{frame}

\section{O Problema de Otimização}

\begin{frame}
\frametitle{O Problema de Otimização com Restrições}
\begin{itemize}
    \item Objetivo: Maximizar ou minimizar uma função $f(x,y)$
    \item Sujeito a uma restrição $g(x,y) = c$
    \item Exemplo: Maximizar a área de um retângulo com perímetro fixo
\end{itemize}
\end{frame}

\section{A Técnica}

\begin{frame}
\frametitle{A Técnica dos Multiplicadores de Lagrange}
\begin{tikzpicture}[
  scale=0.1,
  node distance=.4cm and .3cm,
  box/.style={rectangle, rounded corners, draw=gray, fill=black!80, text width=10cm, minimum height=.3cm, text centered, font=\small\color{white}},
  arrow/.style={draw=black, thick, -{Latex[length=3mm, width=2mm]}}
]

\node[box, font=\bfseries\large\color{cyan}] (title) {Multiplicadores de Lagrange};

\node[box, below=of title] (step1) { Problema: Max/min $f(x, y, \dots)$ sujeito a $g(x, y, \dots) = 0$};
\node[box, below=of step1] (step2) { Ideia: $\nabla f = \lambda \nabla g$ (tangência das curvas de nível)};
\node[box, below=of step2] (step3) {
   Sistema:\\
  $\nabla f(x, y, \dots) = \lambda \nabla g(x, y, \dots)$\\
  $g(x, y, \dots) = 0$
};
\node[box, below=of step3] (step4) { Solução: $(x, y, \dots, \lambda)$ que satisfazem o sistema};
\node[box, below=of step4] (step5) { Interpretação de $\lambda$: Sensibilidade de $f$ em relação à restrição $g$};

% Arrows
\draw[arrow] (title) -- (step1);
\draw[arrow] (step1) -- (step2);
\draw[arrow] (step2) -- (step3);
\draw[arrow] (step3) -- (step4);
\draw[arrow] (step4) -- (step5);
\end{tikzpicture}
\end{frame}

\begin{frame}
\frametitle{A Técnica dos Multiplicadores de Lagrange}
\begin{enumerate}
    \item Definir a função Lagrangiana: $L(x,y,\lambda) = f(x,y) - \lambda(g(x,y) - c)$
    \item Calcular as derivadas parciais e igualá-las a zero:
        \begin{align*}
        \frac{\partial L}{\partial x} &= 0 \\
        \frac{\partial L}{\partial y} &= 0 \\
        \frac{\partial L}{\partial \lambda} &= 0
        \end{align*}
    \item Resolver o sistema de equações
    \item Verificar os pontos críticos encontrados
\end{enumerate}
\end{frame}

\section{Exemplo}

\begin{frame}
\frametitle{Exemplo: Maximização da Área de um Retângulo}
\begin{itemize}
    \item Função objetivo: $f(x,y) = xy$ (área)
    \item Restrição: $g(x,y) = 2x + 2y = P$ (perímetro)
    \item Lagrangiana: $L(x,y,\lambda) = xy - \lambda(2x + 2y - P)$
\end{itemize}
\end{frame}

\begin{frame}
\frametitle{Desenvolvimento do Exemplo: Passo 1}
\large
\begin{block}{Definição do Problema}
\begin{align*}
\text{Maximizar: } & f(x,y) = xy \quad \text{(área do retângulo)} \\
\text{Sujeito a: } & g(x,y) = 2x + 2y = P \quad \text{(perímetro fixo)}
\end{align*}
\end{block}

\begin{block}{Função Lagrangiana}
\begin{align*}
L(x,y,\lambda) &= f(x,y) - \lambda(g(x,y) - P) \\
&= xy - \lambda(2x + 2y - P)
\end{align*}
\end{block}
\end{frame}

\begin{frame}
\frametitle{Desenvolvimento do Exemplo: Passo 2}
\large
\begin{block}{Condições de Primeira Ordem}
Calculamos as derivadas parciais e igualamos a zero:
\begin{align*}
\frac{\partial L}{\partial x} &= y - 2\lambda = 0 \\
\frac{\partial L}{\partial y} &= x - 2\lambda = 0 \\
\frac{\partial L}{\partial \lambda} &= -(2x + 2y - P) = 0
\end{align*}
\end{block}
\end{frame}

\begin{frame}
\frametitle{Desenvolvimento do Exemplo: Passo 3}
\large
\begin{block}{Resolução do Sistema}
Da primeira equação: $y = 2\lambda$ \\
Da segunda equação: $x = 2\lambda$ \\
\vspace{0.3cm}
Portanto: $x = y$ (o retângulo é um quadrado)
\vspace{0.3cm}

Da terceira equação: $2x + 2y = P$ \\
Substituindo $x = y$: $2x + 2x = P$ \\
Simplificando: $4x = P$ \\
Portanto: $x = \frac{P}{4}$
\end{block}

\begin{block}{Solução}
$x = y = \frac{P}{4}$ (quadrado)
\end{block}
\end{frame}

\begin{frame}
\frametitle{Desenvolvimento do Exemplo: Passo 4}
\large
\begin{block}{Análise da Matriz Hessiana}
A matriz Hessiana da Lagrangiana é:
\begin{align*}
H(L) = 
\begin{pmatrix}
\frac{\partial^2 L}{\partial x^2} & \frac{\partial^2 L}{\partial x \partial y} \\
\frac{\partial^2 L}{\partial y \partial x} & \frac{\partial^2 L}{\partial y^2}
\end{pmatrix} =
\begin{pmatrix}
0 & 1 \\
1 & 0
\end{pmatrix}
\end{align*}
\end{block}
\end{frame}


\begin{frame}
\frametitle{Desenvolvimento do Exemplo: Passo 4}
Para verificar as condições de segunda ordem, analisamos o comportamento de $H(L)$ no espaço tangente à restrição.

\begin{block}{Verificação do Ponto Crítico}
\begin{itemize}
\item Determinante: $\det(H) = -1 < 0$, indicando um ponto de sela da Lagrangiana
\item No espaço tangente à restrição: a forma quadrática é definida negativa
\item Conclusão: o ponto $(x,y) = (\frac{P}{4}, \frac{P}{4})$ é um máximo local
\end{itemize}
\end{block}
\end{frame}

\begin{frame}
\frametitle{Desenvolvimento do Exemplo: Visualização}
\begin{center}
\begin{tikzpicture}[scale=0.8]
    % Eixos
    \draw[->] (0,0) -- (6,0) node[right] {$x$};
    \draw[->] (0,0) -- (0,6) node[above] {$y$};
    
    % Curvas de nível da função objetivo (área)
    \draw[blue, thick] plot[domain=0.5:5, samples=100] (\x, {1/\x}) node[right] {$xy = 1$};
    \draw[blue, thick] plot[domain=0.7:5, samples=100] (\x, {2/\x}) node[right] {$xy = 2$};
    \draw[blue, thick] plot[domain=1:5, samples=100] (\x, {4/\x}) node[right] {$xy = 4$};
    \draw[blue, thick] plot[domain=1.5:5, samples=100] (\x, {9/\x}) node[right] {$xy = 9$};
    
    % Restrição (perímetro constante)
    \draw[red, thick] plot[domain=0:4, samples=100] (\x, {(8-2*\x)/2}) node[above] {$2x + 2y = 8$};
    
    % Ponto ótimo
    \filldraw[black] (2,2) circle (3pt) node[above right] {$(P/4, P/4)$};
    
    % Legenda
    \node[blue] at (5,5.5) {Curvas de nível $f(x,y) = xy$};
    \node[red] at (5,5) {Restrição $g(x,y) = 2x + 2y = P$};
\end{tikzpicture}
\end{center}
\end{frame}

\section{Aplicações}

\begin{frame}
\frametitle{Aplicações}
\begin{itemize}
    \item Física: Princípio do trabalho virtual, mecânica analítica
    \item Engenharia: Otimização de design, controle de sistemas
    \item Economia: Maximização de utilidade sob restrições orçamentárias
    \item Aprendizado de máquina: Otimização de funções de perda com regularização
\end{itemize}
\end{frame}

\section{Conclusão}

\begin{frame}
\frametitle{Conclusão}
\begin{itemize}
    \item Os multiplicadores de Lagrange são uma ferramenta versátil para otimização com restrições
    \item Permitem transformar problemas com restrições em problemas sem restrições
    \item Amplamente aplicáveis em diversos campos da ciência e engenharia
    \item Compreender esta técnica é fundamental para abordar problemas complexos de otimização
\end{itemize}
\end{frame}

\end{document}
