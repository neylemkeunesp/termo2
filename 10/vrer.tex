\thispagestyle{headandfoot}
\begin{center} {\large Verifica��o RER}
\end{center}
\vspace{0.5cm} Nome:\rule{14cm}{0.01cm} \\



\vspace{1 cm}
 
{\bf S\'o ser\~ao aceitas respostas devidamente justificadas.}

\vspace{1 cm}
\begin{questions}

\question[2.5] Considere uma fun��o de probabilidade 
localizada em [-1,1] e dada por $q(x)=A(1-|x|)$. Determine:
\begin{itemize}
\item $A$ para que $q(x)$ seja normalizada.
\item O valor m�dio de $x$ e $x^2$. 
\end{itemize}

\question[2.5] Considere um sistema com $N$ estados poss�veis.
Calcule a entropia quando:
\begin{itemize}
\item Um estado possui $p_i=1$.
\item Todos os estados s�o equiprov�veis.
\item A metade dos estados � equiprov�vel e a outra 
metade ocorre com $p_i=0$. Assuma $N$ par. 
\end{itemize}

\question[2.5] Um mole de um g�s de van der Waals � comprimido
de forma quase-est�tica do volume $V_1$ at� o volume $V_2$. 
Para um g�s deste tipo temos que:

$$p=\frac{RT}{V-b}-\frac{a}{V^2}$$

Calcule o trabalho realizado pelo g�s. 
 
 \question[2.5] Considere um sistema cooperativo com energias:
$3\epsilon_o$, $2\epsilon_o$ e 0 com degeneresc�ncias $\gamma$, 2 e 1. 
Determine:
\begin{parts}
\item a fun��o parti��o.
\item a energia livre de Helmholtz
\item a depend�ncia de $U$ com $T$. 
\end{parts}

\end{questions}


%%% Local Variables: 
%%% mode: latex
%%% TeX-master: "exame"
%%% End: 
