\thispagestyle{headandfoot}
\begin{center} {\large Verifica��o IV}
\end{center}
\vspace{0.5cm} Nome:\rule{14cm}{0.01cm} \\



\vspace{1 cm}
 
{\bf S\'o ser\~ao aceitas respostas devidamente justificadas.}

\vspace{1 cm}
\begin{questions}

\question[1.5] Considere um sistema com tr�s estados em $T=200 K$
eles possuem probabilidade de ocupa��o $p_1=0.8$, $p_2=0.1$ e $p_3=0.1$. 
Determine as energias e a fun��o de parti��o do sistema. Considere que $k=1$ 
para simplificar as contas. 


\question [1.5] Calcule os dois n�veis de energia mais baixo para um el�tron em
uma caixa de 1 \AA Calcule a fun��o parti��o em $T=350 K$. 

 \question[1.5] Mol�culas de gases ideias diat�micos possuem $q=300$ em $T=300$. Determine a temperatura a partir da qual $q<10$, nesta escala de temperaturas os efeitos quanticos passam a ser importantes.  

\end{questions}


%%% Local Variables: 
%%% mode: latex
%%% TeX-master: "exame"
%%% End: 
