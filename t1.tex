\thispagestyle{headandfoot}
\begin{center} {\large Trabalho I}
\end{center}
\vspace{0.5cm} Nome:\rule{14cm}{0.01cm} \\



\vspace{1 cm}
 
{\bf S\'o ser\~ao aceitas respostas devidamente justificadas.}

\vspace{1 cm}
\begin{questions}
  
  \question[1.0] Fa\c{c}a uma pesquisa bibliogr�fica e descreva o 
ciclo de Diesel qualitativamente. Descreva algumas de suas aplica\c{c}�es
tecnol�gicas.
\begin{parts}

\item Desenhe o ciclo de Diesel no plano p-V e no plano T-S.
\item Determine a efici�ncia do ciclo em fun\c{c}�o dos volumes e press�es
envolvidos.
\item Compare este resultado com o resultado do ciclo de Otto. 
\end{parts}

\question[1.0] Um g�s de f�tons obedece as seguintes equa\c{c}�es:
$$U=bVT^4$$
 e
$$P=\frac{U}{3V}$$
\begin{parts}

\item Determine $S=S(U,V)$.
\item Determine $F=F(T,V)$.
\item Determine $C_V$ e $C_P$. 
\end{parts}

\end{questions}



%%% Local Variables: 
%%% mode: latex
%%% TeX-master: "exame"
%%% TeX-master: "exame"
%%% TeX-master: "exame"
%%% TeX-master: "exame"
%%% End: 
