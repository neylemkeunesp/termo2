\thispagestyle{headandfoot}
\begin{center} {\large Prova Substitutiva}
\end{center}
\vspace{0.5cm} Nome:\rule{14cm}{0.01cm} \\



\vspace{1 cm}
 
{\bf S\'o ser\~ao aceitas respostas devidamente justificadas.}

\vspace{1 cm}
\begin{questions}
  \question[2.0] Reduza as derivadas a express�es que contenham
  $\alpha$, $\kappa_T$ e $C_p$.
  \begin{parts}
  \item $$\left( \frac{\partial T}{\partial S} \right)_p$$
  \item $$\left( \frac{\partial T}{\partial p} \right)_H$$
  \end{parts}

%  \question[2.0] Calcule a transformada de Legendre de $f(x)=|x|^3$,
%  esboce o gr�fico da fun��o e de sua transformada.

  \question[2.0] Demonstre a rela��o de Slater
  \[
  \gamma_G=-\frac{1}{6}-\frac{v}{2 B_S}\frac{\partial B_S}{\partial v}
  \]
  a partir da rela��o entre $v_{som}$ e o m�dulo da
  elasticidade volum�trica $B_S$.  

% \question[1.5] Considere a
%   equa��o de Berthelot:
% $$ \left(p+\frac{a}{v^2 T}\right)(v-b)=RT$$
% \begin{parts}
% \item Determine em fun��o de $a$, $b$ e $R$ os valores da
%   press�o, volume molar e temperatura cr�tica.
% \item Escreva a equa��o de Berthelot em termos da press�o,
%   volume molar e temperatura cr�tica.
% \item Nas proximidades de $T_c$ temos que
% $$p=p_o+A(v-v_o)$$
% Determine o valor de $A$, $v_o$ obedece $p^{\prime\prime}(v_o)=0$ e
% $p_o=p(v_o)$.
% \end{parts}

\question[2.0] Dois corpos identicos possuem temperatura $T_1$ e
$T_2$.  Eles s�o colocados em contato t�rmico e atingem o estado
de equil�brio.  Ache o trabalho m�ximo que pode ser extra�do
destes corpos e a temperatura de equil�brio. Suponha $C_V$ constante. 



\question[2.0] Considere o Hamiltoniano dado por:
$$H=J\sum_{i}^N \sigma_j^2$$ 
onde $\sigma_j=-2,0,2$.
\begin{parts}
\item Utilizando o ensemble microcan�nico determine a entropia $S$
  e o calor espec�fico a volume constante.
\item Repita os mesmos c�lculos utilizando o ensemble can�nico.
\end{parts}

\question[2.0] Considere o Hamiltoniano dado por:
$$H=\sum_{i}^N \frac{p_i^2}{2 m}+\frac{1}{2}m \omega^2 |x_i|$$ 
que representa part�culas em uma dimens�o.  Utilizando o
ensemble can�nico determine a energia livre de Helmholtz e o calor
espec�fico a volume constante. Discuta a validade do princ�pio
de equiparti��o de energia para este caso.



\end{questions}


%%% Local Variables: 
%%% mode: latex
%%% TeX-master: "exame"
%%% End: 
