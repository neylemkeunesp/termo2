\documentclass[fleqn,a4paper]{exam}
\addtolength{\headheight}{1\baselineskip}
\pointsinmargin
\usepackage[utf8]{inputenc}
\usepackage[portuges]{babel}
\usepackage{amsmath}
\usepackage{amssymb}
\usepackage{theorem}
\usepackage{float}
\usepackage{graphicx}
\widowpenalty=10000
\clubpenalty=10000
\raggedbottom
\newtheorem{definition}{Def.}
\theoremstyle{break} \newtheorem{exemplo}{Exemplo}
\theoremstyle{break} \newtheorem{exercicio}{Exercício}
\newcommand{\ao}{\~{a}o}
\newcommand{\cao}{\c{c}\~{a}o}
\newcommand{\coes}{\c{c}\~{o}es}
\newcommand{\cc}{\c{c}}
\newcommand{\ii}{\'{\i}}
\renewcommand{\sectionmark}[1]{\markboth{#1}{}}
\renewcommand{\subsectionmark}[1]{\markright{#1}}
\lfoot{{\small  DISTRITO DE RUBIÃO JR. S/N C.P. 510 \\
CEP  18618-000-BOTUCATU-SP }}
\rfoot{}
\cfoot{}
\lhead{{\large UNIVERSIDADE JÚLIO DE MESQUITA FILHO-UNESP} \\
INSTITUTO DE BIOCIÊNCIAS DE BOTUCATU \\
DEPARTAMENTO DE FÍSICA E BIOFÍSICA 
}
\pagestyle{empty}
\begin{document}

\begin{center}
{\large Segunda Verificação}

{\large Mecânica Estatística}
\end{center}
\vspace{0.5cm} Nome:\rule{14cm}{0.01cm} \\

\vspace{1 cm}
{\bf Só serão aceitas respostas devidamente justificadas.}
\vspace{1 cm}

\begin{questions}

% --- Questão sobre função partição e níveis de energia (baseada em gasessolidos.tex) ---
\question[3.0] Considere uma molécula diatômica com energia vibracional dada por $\epsilon_v = h\nu(v+\frac{1}{2})$, onde $v = 0, 1, 2, \ldots$ e $\nu$ é a frequência de vibração.
\begin{parts}
  \item Derive a expressão para a função de partição vibracional $q_{vib}$.
  \item Para uma molécula de CO, a frequência vibracional é $\nu = 6.42 \times 10^{13}$ Hz. Calcule a função de partição vibracional a $T = 300$ K e a $T = 1000$ K. Considere $h = 6.626 \times 10^{-34}$ J·s e $k_B = 1.381 \times 10^{-23}$ J/K.
  \item Explique por que a contribuição vibracional para a capacidade térmica é desprezível a baixas temperaturas, mas se torna significativa a altas temperaturas.
\end{parts}

% --- Questão sobre calor específico e flutuações (baseada em caloresp.tex) ---
\question[3.0] O calor específico está relacionado às flutuações de energia em um sistema em contato com um banho térmico.
\begin{parts}
  \item Derive a relação entre o calor específico a volume constante $C_V$ e as flutuações quadráticas médias de energia $\langle(E-U)^2\rangle$, onde $U = \langle E \rangle$ é a energia média.
  \item Explique por que, no limite termodinâmico ($N \to \infty$), as flutuações relativas de energia $\sigma/U$ se tornam desprezíveis. Qual é a dependência dessas flutuações relativas com o número de partículas $N$?
\end{parts}

% --- Questão sobre equipartição de energia (baseada em gasessolidos.tex) ---
\question[3.0] O princípio da equipartição de energia é fundamental para entender o comportamento térmico de sistemas termodinâmicos.
\begin{parts}
  \item Enuncie o princípio da equipartição de energia e explique por que ele não é válido para modos vibracionais em baixas temperaturas.
  \item Calcule o calor específico molar a volume constante ($C_V$) para: (i) um gás monoatômico ideal, (ii) um gás diatômico considerando apenas os graus de liberdade translacionais e rotacionais (sem contribuição vibracional). Explique a diferença entre os resultados.
\end{parts}

\end{questions}

\end{document}
