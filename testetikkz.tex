% Slide 6: Visualização do Exemplo 1
\begin{frame}
\frametitle{Visualização do Exemplo 1}
\begin{tikzpicture}[scale=0.6, transform shape]
% Estilos
\tikzset{
    %box/.style={rectangle, draw, thick, minimum height=1.5cm, text width=3.5cm, align=center, font=\small},
    %population/.style={rectangle, draw, thick, minimum height=5.0cm, minimum width=12cm, fill=blue!10, draw=blue!50},
    %disease/.style={rectangle, draw, thick, fill=red!30, draw=red!50},
    %no_disease/.style={rectangle, draw, thick, fill=green!30, draw=green!50},
    %true_positive/.style={rectangle, draw, thick, fill=red!60, draw=red!80},
    %false_negative/.style={rectangle, draw, thick, fill=red!20, draw=red!40},
    %false_positive/.style={rectangle, draw, thick, fill=green!50, draw=green!70},
    %true_negative/.style={rectangle, draw, thick, fill=green!20, draw=green!40}
}

% Parâmetros
\def\prevalencia{0.01}
\def\sensibilidade{0.95}
\def\especificidade{0.90}
\def\larguraTotal{12}
\def\alturaTotal{5}

% Cálculos derivados
\pgfmathsetmacro{\doenca}{\prevalencia}
\pgfmathsetmacro{\semDoenca}{1-\prevalencia}
\pgfmathsetmacro{\larguraDoenca}{\larguraTotal*\doenca}
\pgfmathsetmacro{\larguraSemDoenca}{\larguraTotal*\semDoenca}
\pgfmathsetmacro{\larguraVP}{\larguraDoenca*\sensibilidade}
\pgfmathsetmacro{\larguraFN}{\larguraDoenca*(1-\sensibilidade)}
\pgfmathsetmacro{\larguraVN}{\larguraSemDoenca*\especificidade}
\pgfmathsetmacro{\larguraFP}{\larguraSemDoenca*(1-\especificidade)}

% População total
\node[population] at (\larguraTotal/2, \alturaTotal/2) {};

% Com doença
\node[disease, minimum width=\larguraDoenca cm, minimum height=\alturaTotal cm, anchor=west] at (0,0) {};
\node[true_positive, minimum width=\larguraVP cm, minimum height=\alturaTotal/2 cm, anchor=north west] at (0,\alturaTotal) {};
\node[false_negative, minimum width=\larguraFN cm, minimum height=\alturaTotal/2 cm, anchor=south west] at (\larguraVP,0) {};

% Sem doença
\node[no_disease, minimum width=\larguraSemDoenca cm, minimum height=\alturaTotal cm, anchor=west] at (\larguraDoenca,0) {};
\node[false_positive, minimum width=\larguraFP cm, minimum height=\alturaTotal/2 cm, anchor=north west] at (\larguraDoenca,\alturaTotal) {};
\node[true_negative, minimum width=\larguraVN cm, minimum height=\alturaTotal/2 cm, anchor=south west] at (\larguraDoenca+\larguraFP,0) {};

%% Anotações
\node[font=\bfseries] at (6,6.2) {Visualização do Teorema de Bayes: Teste Médico};
\node[font=\small] at (6,5.5) {Prevalência: 1\%, Sensibilidade: 95\%, Especificidade: 90\%};

\end{tikzpicture}
\end{frame}