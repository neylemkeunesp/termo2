\documentclass[fleqn,a4paper]{article}
\usepackage[utf8]{inputenc}
\usepackage[portuges]{babel}
\usepackage{amsmath}
\usepackage{amssymb}
\usepackage{graphicx}

\newcommand{\D}{\displaystyle}

% Para evitar orfaos em fim de pagina
\widowpenalty=10000
\clubpenalty=10000
\raggedbottom

\title{Resolução - Primeira Verificação}
\author{Mecânica Estatística}
\date{}

\begin{document}

\maketitle

\section*{Questão 1}
O núcleo de um átomo de hidrogênio (um próton) possui um momento magnético.
Em um campo magnético, o próton possui dois estados de diferentes energias, spin para cima e spin para baixo. Esta é a base da ressonância magnética nuclear para prótons. As populações relativas são dadas pela distribuição de Boltzmann. A diferença de energia entre os dois estados é $\Delta\epsilon =g\mu_B$ onde $g=2.79$ e $\mu=5.05\times 10^{-24}$ J Tesla$^{-1}$. Para um instrumento de ressonância nuclear, $B$=7 Tesla. Considere $T=300$ K.

\subsection*{(a) Calcule a função partição.}

\textbf{Solução:}
A função de partição para um sistema com dois estados é dada por:
\begin{align}
Z = \sum_i e^{-\beta \epsilon_i} = e^{-\beta \epsilon_+} + e^{-\beta \epsilon_-}
\end{align}

Onde $\beta = \frac{1}{k_B T}$, $k_B$ é a constante de Boltzmann ($k_B = 1.38 \times 10^{-23}$ J/K), e $\epsilon_+$ e $\epsilon_-$ são as energias dos estados de spin para cima e spin para baixo, respectivamente.

Podemos definir a diferença de energia como $\Delta\epsilon = \epsilon_+ - \epsilon_- = g\mu B$. Se escolhermos $\epsilon_- = 0$ como referência, então $\epsilon_+ = \Delta\epsilon$.

Calculando o valor de $\Delta\epsilon$:
\begin{align}
\Delta\epsilon &= g\mu B \\
&= 2.79 \times 5.05 \times 10^{-24} \text{ J Tesla}^{-1} \times 7 \text{ Tesla} \\
&= 2.79 \times 5.05 \times 7 \times 10^{-24} \text{ J} \\
&= 98.73 \times 10^{-24} \text{ J} \\
&\approx 9.87 \times 10^{-23} \text{ J}
\end{align}

Agora, calculamos $\beta$:
\begin{align}
\beta &= \frac{1}{k_B T} \\
&= \frac{1}{1.38 \times 10^{-23} \text{ J/K} \times 300 \text{ K}} \\
&= \frac{1}{4.14 \times 10^{-21} \text{ J}} \\
&\approx 2.42 \times 10^{20} \text{ J}^{-1}
\end{align}

Portanto, a função de partição é:
\begin{align}
Z &= e^{-\beta \epsilon_+} + e^{-\beta \epsilon_-} \\
&= e^{-\beta \Delta\epsilon} + e^{0} \\
&= e^{-2.42 \times 10^{20} \text{ J}^{-1} \times 9.87 \times 10^{-23} \text{ J}} + 1 \\
&= e^{-2.39 \times 10^{-2}} + 1 \\
&= 0.9764 + 1 \\
&= 1.9764
\end{align}

\subsection*{(b) Calcule a diferença populacional: $\frac{|N_+-N_-|}{N_++N_-}$}

\textbf{Solução:}
As populações relativas dos estados são dadas pela distribuição de Boltzmann:
\begin{align}
\frac{N_+}{N_-} = \frac{e^{-\beta \epsilon_+}}{e^{-\beta \epsilon_-}} = e^{-\beta \Delta\epsilon} = e^{-2.39 \times 10^{-2}} \approx 0.9764
\end{align}

Sabemos que $N_+ + N_- = N$ (número total de prótons). Podemos escrever:
\begin{align}
\frac{N_+}{N_-} &= 0.9764 \\
\Rightarrow N_+ &= 0.9764 \times N_-
\end{align}

Também temos:
\begin{align}
N_+ + N_- &= N \\
0.9764 \times N_- + N_- &= N \\
N_-(0.9764 + 1) &= N \\
N_- &= \frac{N}{1.9764} \\
N_- &\approx 0.5060 \times N
\end{align}

E, consequentemente:
\begin{align}
N_+ &= 0.9764 \times N_- \\
&= 0.9764 \times 0.5060 \times N \\
&\approx 0.4940 \times N
\end{align}

Agora, calculamos a diferença populacional:
\begin{align}
\frac{|N_+ - N_-|}{N_+ + N_-} &= \frac{|0.4940N - 0.5060N|}{0.4940N + 0.5060N} \\
&= \frac{0.0120N}{N} \\
&= 0.0120 \\
&\approx 0.012 \text{ ou } 1.2\%
\end{align}

\section*{Questão 2}
Dois dados são lançados. Qual é a probabilidade condicional de que a soma seja par, dado que a soma é maior ou igual que 8? Justifique detalhadamente.

\textbf{Solução:}
Para resolver este problema, precisamos calcular a probabilidade condicional $P(A|B)$, onde:
\begin{align}
A &= \{\text{a soma dos dados é par}\} \\
B &= \{\text{a soma dos dados é maior ou igual a 8}\}
\end{align}

A probabilidade condicional é dada por:
\begin{align}
P(A|B) = \frac{P(A \cap B)}{P(B)}
\end{align}

Primeiro, vamos listar todos os possíveis resultados ao lançar dois dados:
\begin{align}
\Omega = \{(1,1), (1,2), \ldots, (6,6)\}
\end{align}

Há um total de $6 \times 6 = 36$ resultados possíveis, todos com a mesma probabilidade de $\frac{1}{36}$.

Agora, vamos identificar os eventos $A$, $B$ e $A \cap B$:

Evento $A$ (soma par): $(1,1), (1,3), (1,5), (2,2), (2,4), (2,6), (3,1), (3,3), (3,5), (4,2), (4,4), (4,6), (5,1), (5,3), (5,5), (6,2), (6,4), (6,6)$

Há 18 resultados com soma par, então $P(A) = \frac{18}{36} = \frac{1}{2}$.

Evento $B$ (soma $\geq 8$): $(2,6), (3,5), (3,6), (4,4), (4,5), (4,6), (5,3), (5,4), (5,5), (5,6), (6,2), (6,3), (6,4), (6,5), (6,6)$

Há 15 resultados com soma maior ou igual a 8, então $P(B) = \frac{15}{36} = \frac{5}{12}$.

Evento $A \cap B$ (soma par e $\geq 8$): $(2,6), (3,5), (4,4), (4,6), (5,3), (5,5), (6,2), (6,4), (6,6)$

Há 9 resultados com soma par e maior ou igual a 8, então $P(A \cap B) = \frac{9}{36} = \frac{1}{4}$.

Portanto, a probabilidade condicional é:
\begin{align}
P(A|B) &= \frac{P(A \cap B)}{P(B)} \\
&= \frac{\frac{1}{4}}{\frac{5}{12}} \\
&= \frac{1}{4} \times \frac{12}{5} \\
&= \frac{3}{5} \\
&= 0.6 \text{ ou } 60\%
\end{align}

\section*{Questão 3}
Quantos subconjuntos de 3 elementos podem ser formados a partir de um conjunto com 8 elementos? E quantas listas ordenadas de 3 elementos distintos podem ser formadas? Justifique.

\textbf{Solução:}
Para calcular o número de subconjuntos de 3 elementos que podem ser formados a partir de um conjunto com 8 elementos, usamos a fórmula de combinação:
\begin{align}
C(n,k) = \binom{n}{k} = \frac{n!}{k!(n-k)!}
\end{align}

Onde $n$ é o número total de elementos e $k$ é o tamanho do subconjunto.

No nosso caso, $n = 8$ e $k = 3$:
\begin{align}
C(8,3) &= \binom{8}{3} = \frac{8!}{3!(8-3)!} = \frac{8!}{3!5!} \\
&= \frac{8 \times 7 \times 6 \times 5!}{3 \times 2 \times 1 \times 5!} \\
&= \frac{8 \times 7 \times 6}{3 \times 2 \times 1} \\
&= \frac{336}{6} \\
&= 56
\end{align}

Portanto, podem ser formados 56 subconjuntos de 3 elementos a partir de um conjunto com 8 elementos.

Para calcular o número de listas ordenadas de 3 elementos distintos, usamos a fórmula de arranjo:
\begin{align}
A(n,k) = \frac{n!}{(n-k)!} = n \times (n-1) \times \cdots \times (n-k+1)
\end{align}

No nosso caso, $n = 8$ e $k = 3$:
\begin{align}
A(8,3) &= \frac{8!}{(8-3)!} = \frac{8!}{5!} \\
&= 8 \times 7 \times 6 \times \frac{5!}{5!} \\
&= 8 \times 7 \times 6 \\
&= 336
\end{align}

Portanto, podem ser formadas 336 listas ordenadas de 3 elementos distintos a partir de um conjunto com 8 elementos.

\section*{Questão 4}
Um caminhante aleatório dá 10 passos, cada passo pode ser para a direita (probabilidade $p=0,6$) ou para a esquerda ($q=0,4$). Qual a probabilidade de ele dar exatamente 7 passos para a direita? Use a fórmula apropriada e calcule o valor.

\textbf{Solução:}
Este é um problema de distribuição binomial. A probabilidade de obter exatamente $k$ sucessos em $n$ tentativas independentes, onde a probabilidade de sucesso em cada tentativa é $p$, é dada por:
\begin{align}
P(X = k) = \binom{n}{k} p^k (1-p)^{n-k}
\end{align}

No nosso caso, $n = 10$ (número total de passos), $k = 7$ (número de passos para a direita), $p = 0.6$ (probabilidade de um passo para a direita) e $q = 1 - p = 0.4$ (probabilidade de um passo para a esquerda).

Calculando:
\begin{align}
P(X = 7) &= \binom{10}{7} (0.6)^7 (0.4)^{10-7} \\
&= \binom{10}{7} (0.6)^7 (0.4)^3 \\
&= \frac{10!}{7!(10-7)!} (0.6)^7 (0.4)^3 \\
&= \frac{10!}{7!3!} (0.6)^7 (0.4)^3 \\
&= \frac{10 \times 9 \times 8 \times 7!}{7! \times 3 \times 2 \times 1} (0.6)^7 (0.4)^3 \\
&= \frac{10 \times 9 \times 8}{3 \times 2 \times 1} (0.6)^7 (0.4)^3 \\
&= 120 \times (0.6)^7 \times (0.4)^3 \\
&= 120 \times 0.0280 \times 0.064 \\
&\approx 120 \times 0.00179 \\
&\approx 0.215 \text{ ou } 21.5\%
\end{align}

Portanto, a probabilidade de o caminhante dar exatamente 7 passos para a direita em 10 passos é aproximadamente 0.215 ou 21.5\%.

\section*{Questão 5}
Determine a fórmula para determinar a entropia a partir da função de partição no caso do ensemble canônico.

\textbf{Solução:}
No ensemble canônico, a função de partição $Z$ é definida como:
\begin{align}
Z = \sum_i e^{-\beta \epsilon_i}
\end{align}

Onde $\beta = \frac{1}{k_B T}$, $k_B$ é a constante de Boltzmann, $T$ é a temperatura absoluta, e $\epsilon_i$ são os níveis de energia do sistema.

A energia média (energia interna) $U$ pode ser obtida a partir da função de partição:
\begin{align}
U &= \langle E \rangle = \sum_i \epsilon_i p_i \\
&= \sum_i \epsilon_i \frac{e^{-\beta \epsilon_i}}{Z} \\
&= -\frac{1}{Z} \frac{\partial Z}{\partial \beta} \\
&= -\frac{\partial \ln Z}{\partial \beta}
\end{align}

A entropia $S$ no ensemble canônico é dada por:
\begin{align}
S &= -k_B \sum_i p_i \ln p_i \\
&= -k_B \sum_i \frac{e^{-\beta \epsilon_i}}{Z} \ln \frac{e^{-\beta \epsilon_i}}{Z} \\
&= -k_B \sum_i \frac{e^{-\beta \epsilon_i}}{Z} [-\beta \epsilon_i - \ln Z] \\
&= k_B \beta \sum_i \epsilon_i \frac{e^{-\beta \epsilon_i}}{Z} + k_B \ln Z \sum_i \frac{e^{-\beta \epsilon_i}}{Z} \\
&= k_B \beta U + k_B \ln Z
\end{align}

Portanto, a fórmula para determinar a entropia a partir da função de partição no caso do ensemble canônico é:
\begin{align}
S = k_B \ln Z + k_B \beta U = k_B \ln Z + \frac{U}{T}
\end{align}

Ou, alternativamente:
\begin{align}
S = k_B \ln Z - \frac{\partial \ln Z}{\partial T}
\end{align}

\end{document}
